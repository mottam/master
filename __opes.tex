	% -- opções da classe memoir --
	12pt,                    % tamanho da fonte
	% openright,             % capítulos começam em pág ímpar (insere página vazia caso preciso)
	% twoside,               % para impressão em verso e anverso. Oposto a oneside
	oneside,
	a4paper,                 % tamanho do papel. 
	% -- opções da classe abntex2 --
	chapter=TITLE,           % títulos de capítulos convertidos em letras maiúsculas
	%section=TITLE,          % títulos de seções convertidos em letras maiúsculas
	%subsection=TITLE,       % títulos de subseções convertidos em letras maiúsculas
	%subsubsection=TITLE,    % títulos de subsubseções convertidos em letras maiúsculas
	% -- opções do pacote babel --
	english,                 % idioma adicional para hifenização
	brazil                   % o último idioma é o principal do documento
]{abntex2}


pdftitle={\@title}, 
pdfauthor={\@author},
pdfsubject={\imprimirpreambulo},
pdfcreator={LaTeX with abnTeX2},
pdfkeywords={palavras}{chave}{aqui}, 
colorlinks=true,       		% false: boxed links; true: colored links
linkcolor=blue,          	% color of internal links
citecolor=blue,        		% color of links to bibliography
filecolor=magenta,      	% color of file links
urlcolor=blue,
bookmarksdepth=4
}


	
	
	
	
	
	
	
	


	\section{Motivação}
		Escrever aqui seus textos \cite{bischofberger2011prototyping, puerta2005ui, sangiorgi2010estendendo, sangiorgi2009molic, jacko2012human, ferreira2012agile, nilsson2009design, neil2014mobile}, se preferir é possível quebrar os capítulos em arquivos e usar o comando \textbackslash include\{nome-do-arquivo-sem-a-extenção-tex\}.


	Conclusão do trabalho
